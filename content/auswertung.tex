\section{Auswertung}
\label{sec:Auswertung}

\subsection{Messung Rechteckfunktion}
 Bei der Messung der Rechteckspannung ergaben sich die in Tab.\ref{tab:1}
 enthaltenen Werte. Zur Berechnung von $\text{a_{n,theo}}$ wurde
 Gleichung.\ref{eqn:rechteck} verwendet.

 \begin{table}[h]
   \centering
   \label{tab:1}
   \begin{tabular}{ c c c c c }
     \toprule
    &$\text{Oberschwingung n}$ \\
    &$ \text{a_{n,exp} \, \, in \,\,[V]}$ \\
    &$ \text{\nu \, \, in \, \, [Hz]}$ \\
    &$ \text{a_{n,theo} \, \, in \, \, [V]}$ \\
    &$ \text{prozentuale Abweichung \,\, in \,\, \si{percent}}$ \\
     \midrule
     1&456  & 60000 & 509.25 & 10.46
     3&148  &175000 & 169.76& 12.82
     5&84  &285000  & 101.86& 17.53
     7&54  &400000  &  72.76& 25.78
     9&43.2&510000  &  56.59& 23.66
      11&37.6&620000&  46.59& 18.79
      13&32.0&735000&  39.18& 18.31
      15&30.4& 845000& 33.95& 10.46
      17&26.4 &955000& 29.96& 11.88

     \bottomrule
   \end{tabular}
   \caption{Werte der Rechteckspannung.}
 \end{table}
