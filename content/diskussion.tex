\section{Diskussion}
\label{sec:Diskussion}

Die auftretenden systematischen Abweichungen in der Messung der Fourier-Koeffizienten
der Rechteck-, Dreiecks- und Sägezahnspannung sind durch die Messgeräte zu erklären.
Zum einen besitzt das Gerät an sich eine Abweichung, welche nicht näher bestimmbar ist.
Zum anderen ist das Ablesen durch die Cursorfunktion des Oszilloskops ungenau, da die
Spannungskurve eine Ausdehnung besitzt.
Es ist jedoch zu sagen, dass die prozentualen Abweichungen der Amplituden,  mit höchstens
25.78$\si{\percent}$ bei der Rechteckspannung, 16.24$\si{\percent}$ bei der Dreiecksspannung
und 23.66$\si{\percent}$ bei der Sägezahnspannung, unter Beachtung der eventuellen systematischen
Abweichung realistisch sind.
Bei der Fourier-Synthese bilden die jeweiligen Summenschwingungen schon eine gute
Approximation der gesuchten Spannungen. Trotz allem sind es noch nicht genug Oberschwingungen um
eine genähert gerade Spannungskurve zu erreichen. Stattdessen sind in diesen Bereichen noch Schwingungen
um die erwüschte Gerade Schwingungen zu erkennen. Dies bezeichnet man als Gibbs'sches Phänomen.
%an den Schwingungen in den Bereichen wo bei genug Oberschwingung eine gerade Spannungskurve zu erkennen ist.
